%!TEX TS-program = xelatex

% Шаблон документа LaTeX создан в 2018 году
% Алексеем Подчезерцевым
% В качестве исходных использованы шаблоны
% 	Данилом Фёдоровых (danil@fedorovykh.ru) 
%		https://www.writelatex.com/coursera/latex/5.2.2
%	LaTeX-шаблон для русской кандидатской диссертации и её автореферата.
%		https://github.com/AndreyAkinshin/Russian-Phd-LaTeX-Dissertation-Template

\documentclass[a4paper,14pt]{article}

\input{data/preambular.tex}
\begin{document} % конец преамбулы, начало документа
\input{data/title.tex}
\tableofcontents
\pagebreak

\section{Задание}

Изучение методов математичекого моделирования электрических схем в статическом режиме.
Изучение способов обеспечения ститического режима работы схем методами математического моделирования.

\section{Краткие теоретические сведения}

Транзисторные сглаживающие фильтры.
Уменьшить массогабаритные показатели можно, используя транзисторные СФ, вместо громоздких LC-фильтров. 
Правда выигрыш транзисторных фильтров компенсируется меньшим КПД.
Рассмотрим типичные схемы транзисторных фильтров.
На рис.~\ref{fig:method1} представлена схема наиболее простого транзисторного фильтра.

\begin{figure}[H]
	\centering
	\includegraphics[width=0.5\linewidth]{image/method_1}
	\caption{Простейший транзисторный фильтр}
	\label{fig:method1}
\end{figure}

На коллектор транзистора VT поступает напряжение с выпрямителя с большой амплитудой пульсаций. 
Цепь базы питается через интегрирующую цепь RC. 
Эта цепочка сглаживает пульсации на базе транзистора.
В принципе, эту цепь можно представить, как RC-фильтр.
Чем больше постоянная времени τ = RC, тем меньше пульсации напряжения на базе транзистора.
Ну а поскольку транзистор включен по схеме эмиттерного повторителя, то на выходе напряжение будет повторять напряжение на базе, т. е. пульсации будут столь же малыми, как и на базе.
Емкость конденсатора С может быть в несколько раз меньше (примерно в h21э раз), чем в LC-фильтре, поскольку базовый ток намного меньше выходного тока фильтра, т. е. коллекторного тока транзистора.
Основное достоинство схемы - простота.
А вот недостатков...
Во-первых, противоречивые требования к сопротивлению резистора R - для уменьшения пульсаций следует увеличивать сопротивление, для повышения КПД - уменьшать.
Во-вторых, сильная зависимость параметров от температуры, тока нагрузки, коэффициента передачи тока базы транзистора (h21э).
Обычно резистор подбирают экспериментально.
Несколько иная схема, приведенная на рис.~\ref{fig:method2}.
В такой схеме цепь базы транзистора запитывается от отдельного источника с напряжением, больше входного.
Схема обладает меньшими пульсациями.

\begin{figure}[H]
	\centering
	\includegraphics[width=0.5\linewidth]{image/method_2}
	\caption{Еще одна схема транзисторного СФ}
	\label{fig:method2}
\end{figure}

Поскольку база питается от отдельного источника, сопротивление резистора можно увеличить и, следовательно, уменьшить пульсации выходного напряжения.
Мощность, выделяемая на резисторе R мала, так как ток базы мал.
Тем не менее, этой схеме присущи те же недостатки, что и предыдущей.
Кроме того, в таком фильтре транзистор может войти в насыщение и все пульсации со входа фильтра без ограничений будут передаваться на выход.
В этот режим транзистор войдет, когда напряжение на базе превысит напряжение на коллекторе.
Ниже приведена схема транзисторного СФ, лишенная вышеуказанных недостатков.

\begin{figure}[H]
	\centering
	\includegraphics[width=0.5\linewidth]{image/method_3}
	\caption{Фильтр с делителем напряжения}
	\label{fig:method3}
\end{figure}

\section{Выполнение работы}

На рис. \ref{fig:shem} приведена электрическая схема фильтра.

\begin{figure}[H]
	\centering
	\includegraphics[width=0.7\linewidth]{image/shem}
	\caption{Электрическая схема фильтра}
	\label{fig:shem}
\end{figure}


На рис.~\ref{fig:work2} показаны результаты моделирования статического режима СФ.
При $R_3$ = 3кОм и $R_4$ = 3.7кОм на эмиттере транзистора напряжение равно половине напряжения питания, 6В.

\begin{figure}[H]
	\centering
	\includegraphics[width=0.7\linewidth]{image/work2}
	\caption{Результат моделирования}
	\label{fig:work2}
\end{figure}


\section{Выводы по работе}

В ходе выполнения лабораторной работы были изучены методы математического моделирования электрических схем в статическом режиме, способы обеспечения статического режима работы схем методами математического моделирования, был обеспечен статический режим работы транзисторного фильтра таким образом, что напряжение на эмиттере транзистора стало равным половине напряжения питания – 6В.
\section{Контрольные вопросы}

\begin{enumerate}
	\item Метод Ньютона-Рафсона для расчета статического режима электрических схем.
	
	Пусть на отрезке $[a,b]$ существует единственный корень уравнения: $f(x^*)=0$, 
	а $f'(x)$ существует, непрерывна и отлична от нуля на $[a,b]$. Перепишем уравнение следующим образом: $f(x^k+(x^*-x^k))=0$
	и применим к этому выражению формула Лагранжа:
	$f(x^k)+f'(\bar{x})(x^*-x^k)=0, \;\bar{x} \in [a,b]$
	Заменим $ \bar x$ на $x^k$, а $x^*$ - на $x^{k+1}$ и получим формулу итерационного процесса:
    $f(x^k)+f'(x^k)(x^{k+1}-x^k)=0.$
	Выразим отсюда $x^{k+1}$:
	
	$x^{k+1}=x^k-\frac{f(x^k)}{f'(x^k)}$
	
	\item Условия сходимости метода Ньютона-Рафсона.

	Метод касательных является частным случаем метода простых итераций $x^{k+1} = g(x^k), k = 0,1,...$
	
	Для которого $g(x) = x - \frac{f(x)}{f'(x)}$
	
	Метод простых итераций сходится тогда и только тогда, когда $|g'(x)|\leq q<1$,
	
	Подставим в последнее условие выражение для g(x) и получим условие сходимости метода касательных:
	
	$\dfrac{\left|f'(x)f''(x) \right|}{(f'(x))^2} \leq q < 1 $
	
%	$ lim | V - V^*| = 0$
	
	\item Метод продолжения решения по параметру
		
	Введем систему нелинейных алгебраических уравнений: 
	
	$$I'(V,t) = 0$$
	
	где t – параметр, изменяющийся от 0 до 1, такой, что при t=0 
	
	$$I'(V,0) =0$$
	
	имеет известное решение $V^0$, а при $t=1 I'(V,1)=0$, соответствующее решению системы уравнений.
	
	При этом основное требование заключается в том, чтобы функция $I'(V,t)$ была непрерывной при изменении t от 0 до 1.
	Тогда изменяя параметр $t$ от 0 до 1 и решая для каждого $t$ систему уравнений методом Ньютона-Рафсона можно найти последовательность $V^0, V^1,...,V^*$и получить требуемое решение.
	
	\item Объясните почему напряжение на эмиттере транзистора должно быть равно половине напряжения питания.
	
	Для получения максимального значения амплитуды выходного неискаженного сигнала рекомендуется задавать напряжение коллектор-эмиттер в точке покоя равным половине напряжения питания.
\end{enumerate}

\end{document} % конец документа