%!TEX TS-program = xelatex

% Шаблон документа LaTeX создан в 2018 году
% Алексеем Подчезерцевым
% В качестве исходных использованы шаблоны
% 	Данилом Фёдоровых (danil@fedorovykh.ru) 
%		https://www.writelatex.com/coursera/latex/5.2.2
%	LaTeX-шаблон для русской кандидатской диссертации и её автореферата.
%		https://github.com/AndreyAkinshin/Russian-Phd-LaTeX-Dissertation-Template

\documentclass[a4paper,14pt]{article}
\input{data/preambular.tex}
\begin{document} % конец преамбулы, начало документа
\input{data/title.tex}
\tableofcontents
\pagebreak

\section{Задание}

Определить динамический диапазон схем транзисторного фильтра на частоте 1 кГц, т.е. определить максимальную амплитуду входного синусоидального сигнала, при которой выходной сигнал остается без изменений.

\section{Краткие теоретические сведения}

Временная область.

Временная область удобна при изображении изменений сигнала во времени. 
Мы все знаем, что такое синусоиды. 
Каждая синусоида характеризуется тремя параметрами: амплитудой, начальной фазой и частотой. 
Одна синусоида имеет одну частоту. 
Частота -- это параметр, показывающий как часто сигнал повторяет сам себя. 
Обратным частоте является период. 
Он соответствует продолжительности, которую занимает во времени один период периодического сигнала. 
На графиках показаны две синусоиды с различными частотами и, следовательно, различными периодами.

\section{Выполнение работы}
	
Схема фильтра представлена на рис.~\ref{fig:schema}, неискаженный сигнал на рис.~\ref{fig:wave_1}.
При $V_{in} = 3.5V$ начинают появляться искажения (рис.~\ref{fig:wave_2}).
Таким образом, динамический диапазон фильтра 3.5В.

\begin{figure}[H]
	\centering
	\includegraphics[width=\linewidth]{image/schema}
	\caption{Схема фильтра}
	\label{fig:schema}
\end{figure}

\begin{figure}[H]
	\centering
	\includegraphics[width=\linewidth]{image/wave_1}
	\caption{Временная характеристика при $V_{in} = 0.5V$}
	\label{fig:wave_1}
\end{figure} 

\begin{figure}[H]
	\centering
	\includegraphics[width=\linewidth]{image/wave_2}
	\caption{Временная характеристика при $V_{in} = 3.5V$}
	\label{fig:wave_2}
\end{figure}
\section{Выводы по работе}

В	ходе  выполнения  работы  были  изучены  методы  математического моделирования транзисторного фильтра во временной области. Экспериментальным путем получен динамический диапазон транзисторного фильтра на частоте 1 кГц с напряжением питания 12 В, равный 3.5 В.


\section{Контрольные вопросы}

\begin{enumerate}
	\item Математическая модель схемы во временной области.
	 
	\begin{equation}
		I(V', V, t) = 0
	\end{equation}
	где $I$ - нелинейная вектор-функция, представляющая собой 
	алгебраическую сумму токов в узле;
	$V'$ - вектор производных узловых потенциалов по времени;
	$V$ - вектор узловых потенциалов;
	$t$ - время. 
	
	\item Методы решения систем дифференциальных уравнений.
	
	Существует несколько методов решения систем дифференциальных уравнений.
	Некоторые из них: метод исключения, метод характеристического уравнения (метод Эйлера), усовершенствованный метод Эйлера, метод Рунге-Кутты 4-го порядка.
\end{enumerate}

\end{document} % конец документа