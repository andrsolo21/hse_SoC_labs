%!TEX TS-program = xelatex

% Шаблон документа LaTeX создан в 2018 году
% Алексеем Подчезерцевым
% В качестве исходных использованы шаблоны
% 	Данилом Фёдоровых (danil@fedorovykh.ru) 
%		https://www.writelatex.com/coursera/latex/5.2.2
%	LaTeX-шаблон для русской кандидатской диссертации и её автореферата.
%		https://github.com/AndreyAkinshin/Russian-Phd-LaTeX-Dissertation-Template

\documentclass[a4paper,14pt]{article}

\input{data/preambular.tex}
\begin{document} % конец преамбулы, начало документа
\input{data/title.tex}
\tableofcontents
\pagebreak

\section{Задание}

Обеспечить заданную АЧХ транзисторного фильтра таким образом, чтобы частота среза по уровню $0,7*max(K_u)$ = 30 Гц.

\section{Краткие теоретические сведения}

Частотная область удобна при изображении частотного состава сигналов.
Каждая синусоида, представленная на графике, имеет одну частоту. 
Следовательно, в частотной области каждая синусоида представляется только одной частотной составляющей.
Ее амплитуда (на графике - прямая со стрелкой вверх) в частотной области пропорциональна амплитуде синусоиды во временной области. 
Частота f1 соответствует частоте первой синусоиды, а f2 - второй. 
Чем выше частота синусоиды, тем дальше по оси частот она располагается. 
(Словосочетание «частотная составляющая» для краткости заменяют просто на «частоту», если понятно, что речь идет о составляющей частотного спектра, а не о понятии частоты как таковом).

\section{Выполнение работы}

На рис.~\ref{fig:work_1} показаны требуемые АЧХ и схема фильтра.

Были получены следующие параметры фильтра:
$R_1 = 56 kOmh, R_2 = 7 kOmh, R_3 = 6 kOmh, R_4 = 7.45 kOmh, C_1 = 10 uF, C_2 = 8 uF, C_3 = 2.6 uF, C_4 = 8.75 uF$.

На рис.~\ref{fig:work_2} показана демонстрация работы полученных параметров.

\begin{figure}[H]
	\centering
	\includegraphics[width=0.9\linewidth]{image/work_1}
	\caption{АЧХ и схема фильтра}
	\label{fig:work_1}
\end{figure}

\begin{figure}[H]
	\centering
	\includegraphics[width=0.9\linewidth]{image/work_2}
	\caption{Полученные параметры фильтра}
	\label{fig:work_2}
\end{figure}

\section{Выводы по работе}

В ходе выполнения лабораторной работы были изучены методы математического моделирования электрических схем в частотной области и способы обеспечения частотных характеристик схем методами математического моделирования.

\section{Контрольные вопросы}

\begin{enumerate}
	\item Математическая модель схемы в частотной области.
	
	В базисе узловых потенциалов математическая модель электрической схемы в частотной области представляет собой систему линейных алгебраических уравнений с комплексными коэффициентами: 
	$$YV=I$$
	где $Y$ -- матрица узловых проводимостей; $V$ -- вектор узловых потенциалов; $I$ -- вектор узловых токов.
	
	Для решения системы линейных уравнений применяют метод $LU$ разложения, в соответствии с которым матрица $Y$ представляется произведением нижней треугольной матрицы с единичной диагональю $L$ и верхней треугольной матрицы $U$:
	$$Y=LU$$
	
	Элементы матриц $L$ и $U$ вычисляются с помощью следующей рекуррентной процедуры:
	
	$$u_{sj} = y_{sj} - \sum_{k=1}^{s-1} l_{sk} u_{kj}, j = s, s+1,..,n;$$
	
	$$l_{is} = \frac{y_{is} - \sum_{k=1}^{s-1} l_{ik} u_{ks}}{u_{ss}}, i = s+1,...,n;$$
	
	После LU разложения матрицы Y, решение системы уравнений заменяется последовательным решением двух систем с треугольными матрицами:
	
	$$ LZ = I; UV=Z$$ 
	
	В результате решения системы уравнений определяется вектор узловых потенциалов, на основе которого рассчитывается комплексный коэффициент передачи, его модуль и фаза:
	
	$$K = \frac{V_j}{V_i}, K = |K| = \sqrt{real^2 K + imag^2 K}, F = arctan \frac{imag K}{real K}$$
	
	\item Методы решения систем линейных алгебраических уравнений.
	
	Есть две группы методов -- прямые и итерационные. 
	
	Прямые методы дают алгоритм, по которому можно найти точное решение систем линейных алгебраических уравнений. Итерационные методы основаны на использовании повторяющегося процесса и позволяют получить решение в результате последовательных приближений.
	
	Некоторые прямые методы:
	
	\begin{itemize}
	\item Метод Гаусса;
	
	\item Метод Крамера;
	
	\item Матричный метод.
	\end{itemize}
	
	Итерационные методы устанавливают процедуру уточнения определённого начального приближения к решению. При выполнении условий сходимости они позволяют достичь любой точности просто повторением итераций. Преимущество этих методов в том, что часто они позволяют достичь решения с заранее заданной точностью быстрее, а также позволяют решать большие системы уравнений. 
	
	
\end{enumerate}

\end{document} % конец документа