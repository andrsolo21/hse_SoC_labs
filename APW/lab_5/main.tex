%!TEX TS-program = xelatex

% Шаблон документа LaTeX создан в 2018 году
% Алексеем Подчезерцевым
% В качестве исходных использованы шаблоны
% 	Данилом Фёдоровых (danil@fedorovykh.ru) 
%		https://www.writelatex.com/coursera/latex/5.2.2
%	LaTeX-шаблон для русской кандидатской диссертации и её автореферата.
%		https://github.com/AndreyAkinshin/Russian-Phd-LaTeX-Dissertation-Template

\documentclass[a4paper,14pt]{article}
\input{data/preambular.tex}
\begin{document} % конец преамбулы, начало документа
\input{data/title.tex}
\tableofcontents
\pagebreak

\section{Задание}

Экспериментально получить вольт – амперную характеристику (ВАХ) полупроводникового диода.
Исследовать влияние температуры на характеристики p-n диодов.


\section{Краткие теоретические сведения}

\subsection{Что такое идеальный диод?}
Основная задача обычного выпрямительного диода – проводить электрический ток в одном направлении, и не пропускать его в обратном. 
Следовательно, идеальный диод должен быть очень хорошим проводником с нулевым сопротивлением при прямом подключении напряжения (плюс - к аноду, минус - к катоду), и абсолютным изолятором с бесконечным сопротивлением при обратном.
Вот так это выглядит на графике:

\begin{figure}[H]
	\centering
	\includegraphics[width=0.5\linewidth]{image/intro_001}
	\caption{График зависимости тока от напряжения на идеальном диоде}
	\label{fig:intro001}
\end{figure}

Такая модель диода используется в случаях, когда важна только логическая функция прибора. Например, в цифровой электронике.

\subsection{ВАХ реального полупроводникового диода}

Однако на практике, в силу своей полупроводниковой структуры, настоящий диод обладает рядом недостатков и ограничений по сравнению с идеальным диодом. 
Это можно увидеть на графике, приведенном ниже.

\begin{figure}[H]
	\centering
	\includegraphics{image/intro_002}
	\caption{Зависимость тока от напряжения в настоящем диоде}
	\label{fig:intro002}
\end{figure}

\subsubsection{$V_{\gamma}$(гамма) -- напряжение порога проводимости.}

При прямом включении напряжение на диоде должно достигнуть определенного порогового значения - $V_{\gamma}$. 
Это напряжение, при котором PN-переход в полупроводнике открывается достаточно, чтобы диод начал хорошо проводить ток. 
До того как напряжение между анодом и катодом достигнет этого значения, диод является очень плохим проводником. 
$V_{\gamma}$ у кремниевых приборов примерно 0.7V, у германиевых – около 0.3V.

\subsubsection{$I_{D\_MAX}$ -- максимальный ток через диод при прямом включении.}

При прямом включении полупроводниковый диод способен выдержать ограниченную силу тока $I_{D\_MAX}$. 
Когда ток через прибор превышает этот предел, диод перегревается. 
В результате разрушается кристаллическая структура полупроводника, и прибор становится непригодным. 
Величина данной силы тока сильно колеблется в зависимости от разных типов диодов и их производителей.

\subsubsection{$I_{OP}$ -- обратный ток утечки.}

При обратном включении диод не является абсолютным изолятором и имеет конечное сопротивление, хоть и очень высокое. 
Это служит причиной образования тока утечки или обратного тока $I_{OP}$. 
Ток утечки у германиевых приборов достигает до 200 µА, у кремниевых до нескольких десятков nА. 
Самые последние высококачественные кремниевые диоды с предельно низким обратным током имеют этот показатель около 0.5 nA.

\subsubsection{$PIV$ (Peak Inverse Voltage) -- Напряжение пробоя.}

При обратном включении диод способен выдерживать ограниченное напряжение – напряжение пробоя $PIV$.
Если внешняя разность потенциалов превышает это значение, диод резко понижает свое сопротивление и превращается в проводник. 
Такой эффект нежелательный, так как диод должен быть хорошим проводником только при прямом включении. 
Величина напряжения пробоя колеблется в зависимости от разных типов диодов и их производителей.

\subsubsection{Паразитическая емкость PN-перехода.}

Даже если на диод подать напряжение значительно выше $V_{\gamma}$, он не начнет мгновенно проводить ток.
Причиной этому является паразитическая емкость PN перехода, на наполнение которой требуется определенное время. 
Это сказывается на частотных характеристиках прибора.

\begin{figure}[H]
	\centering
	\includegraphics[width=0.7\linewidth]{image/intro_003}
	\caption{Паразитическая емкость}
	\label{fig:intro003}
\end{figure}

\subsection{Приближенные модели диодов}

В большинстве случаев, для расчетов в электронных схемах, не используют точную модель диода со всеми его характеристиками. Нелинейность этой функции слишком усложняет задачу. Предпочитают использовать, так называемые, приближенные модели.

\subsubsection{Приближенная модель диода <<идеальный диод + $V_{\gamma}$>>}

Самой простой и часто используемой является приближенная модель первого уровня. Она состоит из идеального диода и, добавленного к нему, напряжения порога проводимости $V_{\gamma}$.

\begin{figure}[H]
	\centering
	\includegraphics[width=0.7\linewidth]{image/intro_004}
	\caption{Приближенная модель диода <<идеальный диод + $V_{\gamma}$>>}
	\label{fig:intro004}
\end{figure}

\subsubsection{Приближенная модель диода <<идеальный диод + $V_{\gamma} + r_D$>>}

Иногда используют чуть более сложную и точную приближенную модель второго уровня. В этом случае добавляют к модели первого уровня внутреннее сопротивление диода, преобразовав его функцию из экспоненты в линейную.

\begin{figure}[H]
	\centering
	\includegraphics[width=0.7\linewidth]{image/intro_005}
	\caption{Приближенная модель диода <<идеальный диод + $V_{\gamma} + r_D$>>}
	\label{fig:intro005}
\end{figure}


\section{Выполнение работы}

На рис.~\ref{fig:temp_25} изображена ВАХ диода при значении тока $I_o = 1.5 uA$.
По информации из таблицы, данный диод -- КД204А.
На рис.~\ref{fig:temp_0}~и~\ref{fig:temp_75} представлены ВАХ диода при $T=0^{\circ} C$ и $T=75^{\circ} C$ соответственно.
При увеличении температуры можно наблюдать, как график ВАХ сдвигается вправо.
Это происходит по причине уменьшения контактной разности потенциалов, возрастания энергии основных носителей заряда, расту диффузионной составляющей тока и увеличению прямого тока. 
	
\begin{figure}[H]
	\centering
	\includegraphics[width=\linewidth]{image/temp_25}
	\caption{ВАХ диода при $T=25^{\circ} C$}
	\label{fig:temp_25}
\end{figure}


\begin{figure}[H]
	\centering
	\includegraphics[width=\linewidth]{image/temp_0}
	\caption{ВАХ диода при $T=0^\circ C$}
	\label{fig:temp_0}
\end{figure}

\begin{figure}[H]
	\centering
	\includegraphics[width=\linewidth]{image/temp_75}
	\caption{ВАХ диода при $T=75^\circ C$}
	\label{fig:temp_75}
\end{figure}
\section{Выводы по работе}

В ходе выполнения лабораторной работы было получено экспериментальным путем ВАХ полупроводникового диода. Найдено значение обратного тока, при котором возможно совпадение графиков ВАХ экспериментального и рассматриваемого диода. Исследовано влияние температуры на вольтамперные характеристики диодов.

\section{Контрольные вопросы}

\begin{enumerate}
	\item Что такое полупроводниковый диод.
	
	Полупроводниковый диод -- это полупроводниковый прибор, во внутренней структуре которого сформирован один p-n-переход.
	
	\item Влияние температуры на характеристики p-n диодов.
	
	При большей температуре p-n-перехода тот же прямой ток достигается при меньшем смещении.
	
	\item Способ снятия ВАХ диодов с помощью амперметра и вольтметра.
	
	Вольтметр подключается параллельно диоду, а амперметр -- последовательно.
	
	\item Работа p-n перехода при прямом и обратном включении.
	
	При прямом включении p-n-перехода внешнее напряжение создает в переходе поле, которое противоположно по направлению внутреннему диффузионному полю. 
	Напряженность результирующего поля падает, что сопровождается сужением запирающего слоя. 
	В результате этого большое количество основных носителей зарядов получает возможность диффузионно переходить в соседнюю область. 
	Диффузионный ток зависит от высоты потенциального барьера и по мере его снижения увеличивается экспоненциально. 
	Повышенная диффузия носителей зарядов через переход приводит к повышению концентрации дырок в области n-типа и электронов в области p-типа.
	Такое повышение концентрации неосновных носителей вследствие влияния внешнего напряжения, приложенного к переходу, называется инжекцией неосновных носителей. 
	Неравновесные неосновные носители диффундируют вглубь полупроводника и нарушают его электронейтральность. 
	Восстановление нейтрального состояния полупроводника происходит за счет поступления носителей зарядов от внешнего источника. 
	Это является причиной возникновения тока во внешней цепи, называемого прямым.
	
	При включении p-n-перехода в обратном направлении внешнее обратное напряжение создает электрическое поле, совпадающее по направлению с диффузионным, что приводит к росту потенциального барьера и увеличению ширины запирающего слоя. 
	Все это уменьшает диффузионные токи основных носителей. 
	Для неосновных носителей поле в p-n-переходе остается ускоряющим, и поэтому дрейфовый ток не изменяется. Таким образом, через переход будет протекать результирующий ток, определяемый в основном током дрейфа неосновных носителей. 
	Поскольку количество дрейфующих неосновных носителей не зависит от приложенного напряжения (оно влияет только на их скорость), то при увеличении обратного напряжения ток через переход стремится к предельному значению $I_S$, которое называется током насыщения.

	\item Основные параметры диода.
	
	\begin{itemize}
      	\item $U_{obr.max.}$ -- максимально-допустимое постоянное обратное напряжение диода;
		
		\item $U_{obr.i.max.}$ -- максимально-допустимое импульсное обратное напряжение диода;
		
		\item $I_{f.max.}$ -- максимальный средний прямой ток за период;
		
		\item $I_{f.i.max.}$ -- максимальный импульсный прямой ток за период;
		
		\item $I_{over.max.}$ -- ток перегрузки выпрямительного диода;
		
		\item $f_{max}$ -- максимально-допустимая частота переключения диода;
		
		\item $f_{work}$ -- рабочая частота переключения диода;
		
		\item $U_{f} | I_{f}$ -- постоянное прямое напряжения диода при токе Iпр;
		
		\item $I_{obr}$ -- постоянный обратный ток диода;
		
		\item $T_{k.max.}$ -- максимально-допустимая температура корпуса диода;
		
		\item $T_{p.max.}$ -- максимально-допустимая температура перехода диода.
	\end{itemize}
	
	
	\item ВАХ идеального диода.
	
	\begin{figure}[H]
		\centering
		\includegraphics[width=0.5\linewidth]{image/intro_001}
		\caption{График зависимости тока от напряжения на идеальном диоде}
	\end{figure}
\end{enumerate}

\end{document} % конец документа