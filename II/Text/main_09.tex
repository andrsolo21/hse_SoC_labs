%!TEX TS-program = xelatex

% Шаблон документа LaTeX создан в 2018 году
% Алексеем Подчезерцевым
% В качестве исходных использованы шаблоны
% 	Данилом Фёдоровых (danil@fedorovykh.ru) 
%		https://www.writelatex.com/coursera/latex/5.2.2
%	LaTeX-шаблон для русской кандидатской диссертации и её автореферата.
%		https://github.com/AndreyAkinshin/Russian-Phd-LaTeX-Dissertation-Template

\documentclass[a4paper,14pt]{article}

\input{data/preambular.tex}
\begin{document} % конец преамбулы, начало документа
\input{data/title_09.tex}
%\input{data/title_03.tex}
\tableofcontents
\pagebreak

\section*{Вопросы на которые необходимо ответить}

\begin{enumerate}
\item Какую проблему может решить проект 
\item Кто заказчик проекта
\item Кто потребитель
\item Какая польза будет от проекта
\item Опишите аналоги и их преимущества/недостатки (со ссылками). Чем ваш проект отличается. Его новизна.
\item Приведите общую схему проекта. 
\item Опишите составляющие устройства/разработки.
\item Опишите основные шаги для достижения результата проекта.
\item Опишите какая аппаратная платформа нужна для реализации проекта.
\item Оцените затраты на реализацию проекта (материальные/временные).	
\end{enumerate}

\section{Идея проекта}

Уличный компактный датчик загрязнения воздуха, в котором будут следующие сенсоры:
\begin{itemize}
	 \item Загрязнения воздуха;
	 \item Термометр;
	 \item Анемометр;
\end{itemize}

Если идти дальше, то можно добавить возможность выгрузки данных на сервер и генерацию карты загрязнения/температуры/ветра в реальном времени и с сохранением истории.

\section{Какую проблему может решить проект }

Данный проект позволит контролировать загрязнение воздуха каждому человеку. Каждый сможет поставить у себя дома данный датчик. При этом всем будет доступна достоверная информация о загрязнении окружающей среды на данной местности.

Также будет составлена более детальная карта распространения тепла и ветра на местности. 

При сопоставлении всех данных вместе можно делать более детальный и качественные выводы о причинах и источниках загрязнения. Например, если сопоставить направление ветра и концентрацию загрязнений в воздухе можно примерно определить источник этого загрязнения, будь то завод или автострада. То же и в сопоставлении температуры и ветра: если по направлению ветра наблюдается резкое изменение температуры, значит там что-то греет воздух, и с определенной долей вероятности там же и будет происходить загрязнение воздуха.

Есть и другие задачи которые данный проект поможет решить. Если человек собирается в гости в другую часть города или вообще в другой город, он может тщательнее изучить куда он направляется и что лучше брать с собой.

Более того, если люди решат переехать, этот проект поможет дать более детальную информацию о месте. О загрязнении и ветре. При прочих равных, человек скорее выберет тот дом, где воздух чище и где при выходе их дома в лицо не будет дуть холодный пронизывающий ветер.

В дальнейшем развитии проекта можно добавить датчик аллергенов. Он будет очень полезен аллергикам. И этот фактор будет будет очень важен для выбора места отдыха или места жительства.

\section{ Кто заказчик проекта}

Заказчиком данного проекта данного проекта может быть компания, которая занимается производством различных измерительных приборов, т.к. эта компания и та сама производит часть необходимых в проекте датчиков.

Также заказчиком может выступать информационная или IT компания. Данные с датчиков должен кто-то собирать, анализировать и предоставлять пользователям, эти задачи как-раз подходят IT компаниям, у которых есть для этого все необходимые ресурсы. Доход IT компаний в этом случае может быть в виде части стоимости датчика и как реклама на сайте.

Еще одним из заказчиков может выступать природоохранная организация. В этом случае компания заинтересована в развити проекта и в сохранении окружающей среды.

\section{ Кто потребитель}

Можно выделить несколько потенциальный потребителей:

\begin{itemize}
	\item Первый из них это, прежде всего, люди, которым хотелось бы жить в чистом городе и дышать чистым воздухом. 
	
	\item Следующий потенциальный потребитель - это природоохранные организации, данные датчики помогут им задокументировать, а в последствии и доказать и, возможно, наказать виновника загрязнения.
	
	\item Еще одним потенциальным потребителем явлются сами производства, они могут расставить датчики по территории предприятия и обнаруживать источники загрязнений на своей территории.
\end{itemize}

\section{ Какая польза будет от проекта}

Польза и важность проекта неоднократно упоминались ранее, распишем ее конкретно и по пунктам:

\begin{itemize}
	\item Контроль за загрязнением воздуха, появляется возможность узнать насколько безопасно сейчас на находится в том или ином месте;
	\item Определение источника загрязнения, с помощью датчиков можно выявить источник загрязнения и предпринять соответствующие меры;
	\item Достоверная информация о погоде в конкретном месте в данное время, при повсеместном использовании датчика данные будут детально описывать погоду в текущий момент;
	\item Справочная информация о регионе/городе/району/улице/дому, позволяет узнать динамику изменения температуры.
\end{itemize}

\section{ Опишите аналоги и их преимущества/недостатки (со ссылками). Чем ваш проект отличается. Его новизна.}

Идея контроля качества воздуха не новая, она хорошо развита для комнатных датчиков. Они отслеживают загрязнение воздуха в помещении и при превышении нормы включают вентиляцию.
\href{https://ventnaz.ru/datchik-kachestva-vozduha-sqa.php?ymclid=15913734219417796388300001}{\textbf{Пример такого датчика.}}
Но этот датчик далек от нашего проекта. Наш датчик должен работать на улице.

Единственный пример похожего проекта можно найти в городе Красноярск.
Там активисты организовали \href{http://air.krasn.ru/map.html}{\textbf{сайт}} по мониторингу экологической ситуации в городе, подробнее можно узнать на их \href{http://air.krasn.ru/map.html}{\textbf{сайте}} и \href{https://www.youtube.com/watch?v=TdJVDiakvSk}{\textbf{из видео на YouTube канале <<Редакция>> }}. Это устройство собирает данные об экологическом состоянии города и отображает их на карте в реальном времени. Безусловно, преимуществом этого проекта является то, что он уже реализован и он работает.
Из недостатков можно выделить то, что это устройство не учитывает направление ветра и то, что все производство локально и не выходит за пределы города.
В интернете не удалось найти примерную стоимость и его точные характеристики.

\section{ Приведите общую схему проекта}

Проект состоит из 2 частей: датчика и сервера.

Рассмотрим из чего состоит датчик. Датчик состоит из 3 сенсоров (загрязнения воздуха, термометра и анемометра), ПЛИС или микрокомпьютер и устройства связи с интернетом (в случае с микрокомпьютером может и не понадобиться, например, в Raspberry Pi есть встроенный модуль wi-fi и Ethernet).
Работает вся эта сборка следующим образом: ПЛИС или микрокомпьютер собирает данные с сенсоров и отправляет их на сервер.

Далее сервер получает эти данные, агрегирует их с данными с других датчиков и отображает их на сайте.

\section{ Опишите составляющие устройства/разработки}

Составляющие устройства (описывались ранее):

\begin{itemize}
	\item ПЛИС или микрокомпьютер;
	\item Загрязнения воздуха;
	\item Термометр;
	\item Анемометр;
\end{itemize}


\section{ Опишите основные шаги для достижения результата проекта}

Основные шаги:

\begin{enumerate}
	\item Написание ТЗ;
	\item Поиск подходящих датчиков;
	\item Создание первого "комнатного" прототипа;
	\item Тестирование первого прототип;
	\item Разработка сервера;
	\item Тестирование совместной работы сервера и датчика;
	\item Тестирование работы датчика на улице;
	\item Итоговое тестирование работы системы;
\end{enumerate}

\section{ Опишите какая аппаратная платформа нужна для реализации проекта}

Аппаратная платформа должна уметь работать с сенсорами, которые используются в нашем устройстве, уметь передавать данные через интернет и обладать небольшими размерами.

Под данное описание подходит микрокомпьютер Raspberry Pi. Он обладает wi-fi модулем, Ethernet модулем, USB портами, имеет небольшой размер, а также у него есть GPIO порты. Будем использовать его.

\section{ Оцените затраты на реализацию проекта (материальные/временные)}

Оценка затрат проекта (временные):

\begin{enumerate}
	\item Написание ТЗ -- 1 неделя; 
	\item Поиск подходящих датчиков -- 1/2 недели;
	\item Создание первого "комнатного" прототипа -- 4 недели;
	\item Тестирование первого прототипа -- 1/2 недели;
	\item Разработка сервера -- 4 недели;
	\item Тестирование совместной работы сервера и датчика -- 1 неделя;
	\item Тестирование работы датчика на улице -- 1 неделя;
	\item Итоговое тестирование работы системы -- 8 недель;
\end{enumerate}

Оценка затрат проекта (временные):

\begin{enumerate}
	\item \href{https://www.chipdip.ru/product/raspberry-pi-3-model-b}{\textbf{Raspberry Pi 3 B+}} -- 4 800 руб.;
	\item Загрязнения воздуха
	\begin{itemize}
		\item \href{https://aliexpress.ru/item/32944660534.html?spm=a2g0v.search0302.3.22.36522105UKLgdF&ws_ab_test=searchweb0_0,searchweb201602_0,searchweb201603_0,ppcSwitch_0&algo_pvid=f8fcd270-0610-4a6f-88f2-d2e5b91afd09&algo_expid=f8fcd270-0610-4a6f-88f2-d2e5b91afd09-3}{\textbf{PMS5003}} -- 892,58 руб.;
		\item \href{https://aliexpress.ru/item/32364918596.html}{\textbf{ZE08-CH2O}} -- 1 013,97 руб.
	\end{itemize}
	\item \href{https://aliexpress.ru/item/32802506308.html?spm=a2g0v.search0302.3.99.619273beIzxcH7&ws_ab_test=searchweb0_0,searchweb201602_0,searchweb201603_0,ppcSwitch_0&algo_pvid=59128224-8253-4e49-9bd7-9480790b1b74&algo_expid=59128224-8253-4e49-9bd7-9480790b1b74-14}{\textbf{Термометр}} -- 32,14 руб.;
	\item \href{https://www.alibaba.com/product-detail/FST200-201-Weather-Station-Anemometer-Wind_60220398084.html}{\textbf{Анемометр}} -- 5000 руб; 
	\item Хостинг для сервера 200 руб./месяц.

\end{enumerate}

В итоге получаем 20 недель (при этом некоторы стадии можно распараллелить) и 11 737 руб. + 200 руб/месяц.

\newpage 
\renewcommand{\refname}{{\normalsize СПИСОК ИСПОЛЬЗОВАННЫХ ИСТОЧНИКОВ}} 
\centering 
\begin{thebibliography}{9} 
	\addcontentsline{toc}{section}{\refname} 
	\bibitem{sql} РЯБЫЧИНА О. П., РЫБАК В. А. МЕТОДЫ И СРЕДСТВА МОНИТОРИНГА ЗАГРЯЗНЕНИЯ АТМОСФЕРНОГО ВОЗДУХА //Проблемы инфокоммуникаций. – 2018. – Т. 1. – №. 1-1. – С. 29-37.
	\bibitem{sql} Титова О. А., Кириллов М. В. ОТСЛЕЖИВАНИЕ СОСТОЯНИЯ АТМОСФЕРНОГО ВОЗДУХА В НАСЕЛЕННЫХ ПУНКТАХ //«СТУДЕНТ: НАУКА, ПРОФЕССИЯ, ЖИЗНЬ» Материалы III всероссийской студенческой научной конференции с международным участием (апрель 2016 г.). – 2016. – С. 363.
	\bibitem{sql}Прокофьев И. В., Азаренко Е. И. Организация непрерывного мониторинга загрязезнения воздуха в Севастополе //Современные технологии: проблемы и перспективы. – 2019. – С. 176-180.
\end{thebibliography}

\end{document} % конец документа