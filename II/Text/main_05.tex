%!TEX TS-program = xelatex

% Шаблон документа LaTeX создан в 2018 году
% Алексеем Подчезерцевым
% В качестве исходных использованы шаблоны
% 	Данилом Фёдоровых (danil@fedorovykh.ru) 
%		https://www.writelatex.com/coursera/latex/5.2.2
%	LaTeX-шаблон для русской кандидатской диссертации и её автореферата.
%		https://github.com/AndreyAkinshin/Russian-Phd-LaTeX-Dissertation-Template

\documentclass[a4paper,14pt]{article}

\input{data/preambular.tex}
\begin{document} % конец преамбулы, начало документа
\input{data/title_05.tex}
\tableofcontents
\pagebreak

\section{Цель работы}

Моделирование работы дешифратора, изучение карт Карно.

\section{Задание}

\begin{enumerate}
\item Выполнить действия, описанный в практической работе 5 -- Часть 1, 2, 3;

\item Оформить отчет, который должен включать: титульный лист, введение и постановку задачи, тему работы, описание всех этапов выполнения проекта, скриншоты (рисунки с подрисуночными подписями) ключевых моментов, выводы;

\item Изменить схему устройства добавив собственный блок памяти.
Объяснить работу устройства;

\item Изменить схему устройства добавив арифметический блок.
Объяснить работу устройства.

\end{enumerate}

\section{Часть 1. «Знакомство со средой проектирования Quartus II. Создание проекта»}

\subsection{Создание проекта}

Для начала необходимо открыть Quartus, открываем его.

Далее заходим в меню File и выбираем New Project Wizard.
В открывшемся окне жмем Next, в следующем выбираем директорию для проекта и вводим имя проекта, жмем 2 раза Next.

В следующем окне выбираем все как на рис. \ref{fig:screenshot42} и жмем далее 2 раза.

\begin{figure}[H]
	\centering
	\includegraphics[width=0.7\linewidth]{image/lab5/Screenshot_42}
	\caption{Создание нового проекта}
	\label{fig:screenshot42}
\end{figure}

На последнем шаге должны получится значения как на рис. \ref{fig:screenshot43}.

\begin{figure}[H]
	\centering
	\includegraphics[width=0.7\linewidth]{image/lab5/Screenshot_43}
	\caption{}
	\label{fig:screenshot43}
\end{figure}

\subsection{Разработка конвейерного умножителя}

\subsubsection{Создание умножителя 8х8 с помощью утилиты Mega Wizard\textregistered  Plug-in Manager}

Для создания необходимо в меню выбрать Tools > Mega Wizard  Plug-in Manager, в открывшемся окне выбираем опцию Create a new custom megafunction variation. Далее в папке Arithmetics выбираем LPM\_MULT (рис. \ref{fig:screenshot412}). Семейство микросхем выбираем Cyclone II.

\begin{figure}[H]
	\centering
	\includegraphics[width=0.7\linewidth]{image/lab5/Screenshot_412}
	\caption{Создание мегафункции}
	\label{fig:screenshot412}
\end{figure}

Поля dataa и datab на рис. \ref{fig:screenshot002} оставляем по 8 бит. Жмем next 2 раза.

\begin{figure}[H]
	\centering
	\includegraphics[width=0.7\linewidth]{image/lab5/screenshot002}
	\caption{Создание мегафункци}
	\label{fig:screenshot002}
\end{figure}

Далее в первом окне выбираем yes и вводим число 2 (рис. \ref{fig:screenshot003}), нажимаем next 2 раза.

\begin{figure}[H]
	\centering
	\includegraphics[width=0.7\linewidth]{image/lab5/screenshot003}
	\caption{Создание мегафункци}
	\label{fig:screenshot003}
\end{figure}

На текущем окне (рис. \ref{fig:screenshot423}) выбираем все в соответствии с рисунком и жмем на finish. После этого мегафункция готова.

\begin{figure}[H]
	\centering
	\includegraphics[width=0.7\linewidth]{image/lab5/Screenshot_423}
	\caption{Создание мегафункци}
	\label{fig:screenshot423}
\end{figure}

\subsubsection{Создание 32х16 RAM с помощью утилиты Mega Wizard\textregistered  Plug-in Manager}

Для создания необходимо в меню выбрать Tools > Mega Wizard  Plug-in Manager, в открывшемся окне выбираем опцию Create a new custom megafunction variation. Далее в папке Memory Compiler выбираем RAM: 2-PORT (рис. \ref{fig:screenshot004}). Семейство микросхем выбираем Cyclone II.

\begin{figure}[H]
	\centering
	\includegraphics[width=0.7\linewidth]{image/lab5/screenshot004}
	\caption{Создание мегафункци RAM}
	\label{fig:screenshot004}
\end{figure}

На следующей странице ничего не нажимаем, жмем только на next.

На странице Widths/Blk Type необходимо установить разрядность входного порта data\_a 16 bit (рис. \ref{fig:screenshot005}). После этого жмем Next 2 раза.

\begin{figure}[H]
	\centering
	\includegraphics[width=0.7\linewidth]{image/lab5/screenshot005}
	\caption{Создание мегафункци RAM}
	\label{fig:screenshot005}
\end{figure}

На странице Regs/Clkens/Aclrs отключем опцию Read output port(s) 'q' (рис. \ref{fig:screenshot006}). Больше ничего не трогаем, жмем Next 2 раза.

\begin{figure}[H]
	\centering
	\includegraphics[width=0.7\linewidth]{image/lab5/screenshot006}
	\caption{Создание мегафункци RAM}
	\label{fig:screenshot006}
\end{figure}

На странице Mem Init выбираем yes и указываем имя файла ram.hex (рис. \ref{fig:screenshot007}). После этого жмем Next.

\begin{figure}[H]
	\centering
	\includegraphics[width=0.7\linewidth]{image/lab5/screenshot007}
	\caption{Создание мегафункци RAM}
	\label{fig:screenshot007}
\end{figure}

Последние шаги совпадают с шагами для создания умножителя.

\subsubsection{Создание HEX файл с помощью редактора Memory Editor}

Вменю File выбираем команду New, в открывшемся окне выбираем Other Files и выбираем Hexadecimal (Intel-Format) File. В открытом окне вводим 32 и 16.

\begin{figure}[H]
	\centering
	\includegraphics[width=0.4\linewidth]{image/lab5/screenshot008}
	\caption{Создание HEX файла}
	\label{fig:screenshot008}
\end{figure}

Получаем следующий файл рис. \ref{fig:screenshot009}

\begin{figure}[H]
	\centering
	\includegraphics[width=0.7\linewidth]{image/lab5/screenshot012}
	\caption{HEX файл}
	\label{fig:screenshot012}
\end{figure}


Заполняем файл при помощи функции Custom Fill Cells.

\subsubsection{Добавление блоков в проект и создание связей}

Создаем схему и подтягиваем туда созданные ранее мегафункции (рис. \ref{fig:screenshot011}).

\begin{figure}[H]
	\centering
	\includegraphics[width=0.8\linewidth]{image/lab5/screenshot011}
	\caption{Итоговая схема}
	\label{fig:screenshot011}
\end{figure}

\section{Часть 2. «Моделирование проекта в среде Quartus II»}

Создадим модуляцию проекта при помощи University Program VWF. 

Добавим пины: clk, wren, data, datab, rdaddres, wraddres, q.

\begin{figure}[H]
	\centering
	\includegraphics[width=0.9\linewidth]{image/lab5/screenshot013}
	\caption{Результат модуляции проекта}
	\label{fig:screenshot013}
\end{figure}

\section{Часть 3. «Компиляция проекта в среде Quartus II. Анализ результатов компиляции»}

\subsection{Компиляция проекта}

Скомпилируем проект, результат компиляции  представлен на рис. \ref{fig:screenshot014}.

\begin{figure}[H]
	\centering
	\includegraphics[width=0.7\linewidth]{image/lab5/screenshot014}
	\caption{Результаты компиляции проекта}
	\label{fig:screenshot014}
\end{figure}

\subsection{RTL представление проекта}

На рис. \ref{fig:screenshot015} RTL представление проекта.

\begin{figure}[H]
	\centering
	\includegraphics[width=0.7\linewidth]{image/lab5/screenshot015}
	\caption{RTL представление}
	\label{fig:screenshot015}
\end{figure}

Внутренности ram (рис. \ref{fig:screenshot016}), внутри он состоит из 16 однобитных регистров.

\begin{figure}[H]
	\centering
	\includegraphics[width=0.7\linewidth]{image/lab5/screenshot016}
	\caption{Внутренности ram}
	\label{fig:screenshot016}
\end{figure}

\subsection{Редактор chip planner}

Нажмем правой кнопкой на элемент однобитного элемента ram, далее Locate Node -> Locate in Chip planner. 
Результат представлен на рис. \ref{fig:screenshot017}.

\begin{figure}[H]
	\centering
	\includegraphics[width=0.7\linewidth]{image/lab5/screenshot017}
	\caption{Отображение связей элемента}
	\label{fig:screenshot017}
\end{figure}

\section{Доп задание}

Изменим схему, вместо первого числа поставим счетчик, а вместо двухпортовой памяти поставим однопортовую (рис. \ref{fig:screenshot018}).

\begin{figure}[H]
	\centering
	\includegraphics[width=0.7\linewidth]{image/lab5/screenshot018}
	\caption{Блок схема доп задания}
	\label{fig:screenshot018}
\end{figure}

Результаты компиляции проекта (рис. \ref{fig:screenshot021})

\begin{figure}[H]
	\centering
	\includegraphics[width=0.7\linewidth]{image/lab5/screenshot021}
	\caption{Результат компиляции доп задания}
	\label{fig:screenshot021}
\end{figure}


Временная диаграмма доп задания (рис. \ref{fig:screenshot019}).

\begin{figure}[H]
	\centering
	\includegraphics[width=0.7\linewidth]{image/lab5/screenshot019}
	\caption{WVF диаграмма}
	\label{fig:screenshot019}
\end{figure}

RTL представление доп задания (рис. \ref{fig:screenshot020}).

\begin{figure}[H]
	\centering
	\includegraphics[width=0.7\linewidth]{image/lab5/screenshot020}
	\caption{RTL представление}
	\label{fig:screenshot020}
\end{figure}

Связи на Chip Planner (рис. \ref{fig:screenshot022}).

\begin{figure}[H]
	\centering
	\includegraphics[width=0.7\linewidth]{image/lab5/screenshot022}
	\caption{Отображение связей элемента}
	\label{fig:screenshot022}
\end{figure}


\section{Вывод}
В ходе проделанной работы было создано 2 проекта арифметических устройств с памятью.

Были созданы мегафункции умножителя, счетчика одно- и дву- портовые элементы памяти при помощи Mega Wizard  Plug-in Manager.

Схема была протестированы при помощи WaveForm, были рассмотрены схемы, полученные при помощи Chip planner.
\newpage 
\renewcommand{\refname}{{\normalsize СПИСОК ИСПОЛЬЗОВАННЫХ ИСТОЧНИКОВ}} 
\centering 
\begin{thebibliography}{9} 
	\addcontentsline{toc}{section}{\refname} 
	\bibitem{sql} Vijayakumar P., Vijayalakshmi V., Zayaraz G. Comparative study of hyperelliptic curve cryptosystem over prime field and its survey //International Journal of Hybrid Information Technology. – 2014. – Т. 7. – №. 1. – С. 137-146.
	\bibitem{sql} Антонов А., Филиппов А., Золотухо Р. Средства системной отладки САПР Quartus II //Компоненты и технологии. – 2008. – №. 89.
\end{thebibliography}

\end{document} % конец документа