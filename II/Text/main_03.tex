%!TEX TS-program = xelatex

% Шаблон документа LaTeX создан в 2018 году
% Алексеем Подчезерцевым
% В качестве исходных использованы шаблоны
% 	Данилом Фёдоровых (danil@fedorovykh.ru) 
%		https://www.writelatex.com/coursera/latex/5.2.2
%	LaTeX-шаблон для русской кандидатской диссертации и её автореферата.
%		https://github.com/AndreyAkinshin/Russian-Phd-LaTeX-Dissertation-Template

\documentclass[a4paper,14pt]{article}

\input{data/preambular.tex}
\begin{document} % конец преамбулы, начало документа
\input{data/title_02.tex}
\tableofcontents
\pagebreak

\section{Цель работы}

Моделирование работы дешифратора, изучение карт Карно.

\section{Задание}

\begin{itemize}
	\item Вариант -- 13;
	
	\item Начало диапазона -- 0xC38;
	
	\item Конец диапазона -- 0xC3E;
	
	\item Исключение -- 0x3CA, 0x3CB;
\end{itemize}

\begin{enumerate}
\item Создать схему для проверки функции дешифратора и произвести замер временных задержек.

\item Запрограммировать учебную плату и продемонстрировать результаты работы на макете.

\item Построить временную диаграмму и выполнить моделирование в режимах Functional и Time. Сравнить и обосновать полученные результаты.
\end{enumerate}



\section{Выполнение работы}

\subsection{Вывод формулы}

Переведем шестнадцатеричные значения в их двоичные преставления

0x3C8 = 0011 1100 1000

0x3CA = 0011 1100 1010

0x3CB = 0011 1100 1011

0x3CE = 0011 1100 1110

У всех значений общая часть первые 9 битов, для остальных трех запишем таблицу истинности (таблица \ref{tab:ist}) и по ней составим карту Карно (таблица \ref{tab:karno}).

\begin{table}[H]
	\caption{Таблица истинности}	
	\label{tab:ist}
	\begin{center}
		\begin{tabular}{|l|l|l|l|}
			\hline
			$X_2$ & $X_1$ & $X_0$ & $Y$ \\ \hline
			0 & 0 & 0 & 1 \\ \hline
			0 & 0 & 1 & 1 \\ \hline
			0 & 1 & 0 & 0 \\ \hline
			0 & 1 & 1 & 0 \\ \hline
			1 & 0 & 0 & 1 \\ \hline
			1 & 0 & 1 & 1 \\ \hline
			1 & 1 & 0 & 1 \\ \hline
			1 & 1 & 1 & 0 \\ \hline
		\end{tabular}
	\end{center}
\end{table}

\begin{table}[H]
	\caption{Карта Карно}	
	\label{tab:karno}
	\begin{center}
		\begin{tabular}{c|c|c|c|c|}
			& \multicolumn{2}{c|}{$X_2$}     & \multicolumn{2}{c|}{$\overline X_2$}    \\ \hline
			$X_1$           & 1                 & 0        & 0             & 0                     \\ \hline
			$\overline X_1$ & 1                 & 1        & 1             & 1                     \\ \hline
			& $\overline X_0$     & \multicolumn{2}{c|}{$X_0$} & $\overline X_0$        
		\end{tabular}
	\end{center}
\end{table}

Получим минимизированную функцию из карты Карно, формула \ref{f:f1}.

\begin{equation}
\label{f:f1}
\begin{gathered}
F_{012} = \overline X_1 + X_2 * \overline X_0 
\end{gathered}
\end{equation}

Итоговый результат дешифратора можно выразить через формулу \ref{f:f2}.

\begin{equation}
\label{f:f2}
\begin{gathered}
F_{itog} = \overline X_{11} * \overline X_{10} * X_9 * X_8 * X_7 * X_6 * \overline X_5 * \overline X_4 * X_3 * (\overline X_1 + X_2 * \overline X_0)
\end{gathered}
\end{equation}

\subsection{Составление программы}

	\begin{figure}[H]
		\centering
		\includegraphics[width=0.7\linewidth]{image/03_bdf}
		\caption{bdf представление итоговой функции дешифратора}
		\label{fig:03bdf}
	\end{figure}

	\begin{figure}[H]
		\centering
		\includegraphics[width=0.7\linewidth]{image/03_foto.JPG}
		\caption{фото рабочей платы}
		\label{fig:03foto}
	\end{figure}

\subsection{Тестирование программы}

	\begin{figure}[H]
		\centering
		\includegraphics[width=0.7\linewidth]{image/03_times}
		\caption{Временные задержки}
		\label{fig:03times}
	\end{figure}

	Временные задержки (рис. \ref{fig:03times}) были получены следующим образом:
	TimeQuest Timing Analysis > Write SDC file.. > Report Datasheet.

	\begin{figure}[H]
		\centering
		\includegraphics[width=1\linewidth]{image/03_wvf}
		\caption{Временная диаграмма Functional}
		\label{fig:03wvf}
	\end{figure}

	\begin{figure}[H]
		\centering
		\includegraphics[width=1\linewidth]{image/03_wvf1}
		\caption{Временная диаграмма Time}
		\label{fig:03wvf1}
	\end{figure}


 Моделирование в режимах Functional и Time не отличается, т.к. на приведенной частоте задержек не видно.

\section{Вывод}
В ходе проделанной работы был создан дешифратор, принимающий сигналы в определенных границах с исключениями.
Для минимизации были использованы карты Карно.
При помощи TimeQuest Timing Analysis были получены задержки для кажого входного параметра.
Схема была протестированы при помощи WaneForm, а также удалось загрузить схему на плату и протестировать работоспособность программы на плате.

\newpage 
\renewcommand{\refname}{{\normalsize СПИСОК ИСПОЛЬЗОВАННЫХ ИСТОЧНИКОВ}} 
\centering 
\begin{thebibliography}{9} 
	\addcontentsline{toc}{section}{\refname} 
	\bibitem{sql} Vijayakumar P., Vijayalakshmi V., Zayaraz G. Comparative study of hyperelliptic curve cryptosystem over prime field and its survey //International Journal of Hybrid Information Technology. – 2014. – Т. 7. – №. 1. – С. 137-146.
	\bibitem{sql} Антонов А., Филиппов А., Золотухо Р. Средства системной отладки САПР Quartus II //Компоненты и технологии. – 2008. – №. 89.
\end{thebibliography}

\end{document} % конец документа