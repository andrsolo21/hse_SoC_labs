%!TEX TS-program = xelatex

% Шаблон документа LaTeX создан в 2018 году
% Алексеем Подчезерцевым
% В качестве исходных использованы шаблоны
% 	Данилом Фёдоровых (danil@fedorovykh.ru) 
%		https://www.writelatex.com/coursera/latex/5.2.2
%	LaTeX-шаблон для русской кандидатской диссертации и её автореферата.
%		https://github.com/AndreyAkinshin/Russian-Phd-LaTeX-Dissertation-Template

\documentclass[a4paper,14pt]{article}

\input{data/preambular.tex}
\begin{document} % конец преамбулы, начало документа
\input{data/title_04.tex}
\tableofcontents
\pagebreak

\section{Цель работы}

Моделирование работы дешифратора, изучение карт Карно.

\section{Задание}

\begin{enumerate}
\item Используя логические элементы спроектировать схему и исследовать работу (снять временную характеристику и таблицу задержек) одноразрядного двухразрядного и четырехразрядного сумматора.

\item Построить временную диаграмму и выполнить моделирование в режимах Functional и Time. Сравнить и обосновать полученные результаты.

\item Запрограммировать учебную плату и продемонстрировать результаты работы на макете.

\item На базе сумматора построить вычитатель. Спроектировать его схему и исследовать работу (снять временную диаграмму и таблицу задержек)

\item Построить временную диаграмму и выполнить моделирование в режимах Functional и Time. Сравнить и обосновать полученные результаты.

\item Запрограммировать учебную плату и продемонстрировать результаты работы на макете.
\end{enumerate}

\section{Выполнение работы}

\subsection{Исследование сумматоров}

\subsubsection{Проектирование одноразрядного сумматора}

Построим схему одноразрядного сумматора (рис. \ref{fig:041bdf}).

\begin{figure}[H]
	\centering
	\includegraphics[width=0.7\linewidth]{image/04_1_bdf}
	\caption{Схема одноразрядного сумматора}
	\label{fig:041bdf}
\end{figure}


Для одноразрядного сумматора получились следующие задержки (рис. \ref{fig:041time}). Временные задержки были получены следующим образом: TimeQuest Timing Analysis > Write SDC file.. > Report Datasheet.

\begin{figure}[H]
	\centering
	\includegraphics[width=0.7\linewidth]{image/04_1_time}
	\caption{Временные задержки одноразрядного сумматора}
	\label{fig:041time}
\end{figure}

В файле с временными диаграммами добавим все пины и установим входные значения. Полученные временные диаграммы рис. \ref{fig:041wvf}.

\begin{figure}[H]
	\centering
	\includegraphics[width=0.9\linewidth]{image/04_1_wvf}
	\caption{Временные диаграммы одноразрядного сумматора}
	\label{fig:041wvf}
\end{figure}

 Моделирование в режимах Functional и Time не отличается, т.к. на приведенной частоте задержек не видно.
 
 Загрузим на плату и протестируем работу (рис. \ref{fig:041foto})
 
 \begin{figure}[H]
 	\centering
 	\includegraphics[width=0.7\linewidth]{image/lab4/2020-03-1412-17-02.JPG}
 	\caption{Фото рабочей платы}
 	\label{fig:041foto}
 \end{figure}

(Фото есть т.к. работа выполнялась до карантина)

Далее соберем данную схему как один логический элемент. Для этого необходимо выбрать в меню File Create/Update, далее Create Symbol Files for Current File (рис. \ref{fig:04createsymbol}).

\begin{figure}[H]
	\centering
	\includegraphics[width=0.7\linewidth]{image/04_create_symbol}
	\caption{Создание логического элемента}
	\label{fig:04createsymbol}
\end{figure}

В открывшемся окне сохраняем элемент.

\subsubsection{Исследование работы двухразрядного сумматора}

В меню логических элементов найдем созданный нами одноразрядный сумматор. На его основе спроектируем двухразрядный сумматор (рис. \ref{fig:042sumbdf}).

\begin{figure}[H]
	\centering
	\includegraphics[width=0.7\linewidth]{image/lab4/04_2sum_bdf}
	\caption{Схема двухразрядного сумматора}
	\label{fig:042sumbdf}
\end{figure}

По описанному ранее порядку найдем временные задержки (рис. \ref{fig:042sumtime}).

\begin{figure}[H]
	\centering
	\includegraphics[width=0.7\linewidth]{image/lab4/04_2sum_time}
	\caption{Временные задержки двухразрядного сумматора}
	\label{fig:042sumtime}
\end{figure}

В результате моделирования временная диаграмма имеет следующий вид (рис. \ref{fig:042sumwvf}).

\begin{figure}[H]
	\centering
	\includegraphics[width=0.9\linewidth]{image/lab4/04_2sum_wvf}
	\caption{Временная диаграмма для двухразрядного сумматора}
	\label{fig:042sumwvf}
\end{figure}

 Моделирование в режимах Functional и Time не отличается, т.к. на приведенной частоте задержек не видно.
 
 Загрузим на плату и протестируем работу (рис. \ref{fig:042foto})
 
 \begin{figure}[H]
 	\centering
 	\includegraphics[width=0.7\linewidth]{image/lab4/2020-03-1412-17-16.JPG}
 	\caption{Фото рабочей платы}
 	\label{fig:042foto}
 \end{figure}

\subsubsection{Исследование работы четырехразрядного сумматора}

В меню логических элементов найдем созданный нами одноразрядный сумматор. На его основе спроектируем четырехразрядный сумматор (рис. \ref{fig:044sumbdf}).

\begin{figure}[H]
	\centering
	\includegraphics[width=0.7\linewidth]{image/lab4/04_4sum_bdf}
	\caption{Схема четырехразрядного сумматора}
	\label{fig:044sumbdf}
\end{figure}

По описанному ранее порядку найдем временные задержки (рис. \ref{fig:044sumtime}).

\begin{figure}[H]
	\centering
	\includegraphics[width=0.7\linewidth]{image/lab4/04_4sum_time}
	\caption{Временные задержки четырехразрядного сумматора}
	\label{fig:044sumtime}
\end{figure}


В результате моделирования временная диаграмма имеет следующий вид (рис. \ref{fig:044sumwvf}).

\begin{figure}[H]
	\centering
	\includegraphics[width=0.9\linewidth]{image/lab4/04_4sum_wvf}
	\caption{Временная диаграмма для четырехразрядного сумматора}
	\label{fig:044sumwvf}
\end{figure}

Моделирование в режимах Functional и Time не отличается, т.к. на приведенной частоте задержек не видно.

Загрузим на плату и протестируем работу (рис. \ref{fig:044foto})

\begin{figure}[H]
	\centering
	\includegraphics[width=0.7\linewidth]{image/lab4/2020-03-1412-17-14.JPG}
	\caption{Фото рабочей платы}
	\label{fig:044foto}
\end{figure}

\subsection{Исследование работы четырехразрядного вычитателя}

Есть 2 числа $A = a_3a_2a_1a_0$ и $B = b_3b_2b_1b_0$, необходимо получить разность чисел $B$ и $A$ $S = s_3s_2s_1s_0$. $p$ - сигнал переполнения.

Для реализации вычитателя необходимо инвертировать вычитаемое число и вход переноса из предыдущего разряда подать логическую единицу.

В меню логических элементов найдем созданный нами одноразрядный сумматор. На его основе спроектируем четырехразрядный вычитатель (рис. \ref{fig:044vicbdf}).

\begin{figure}[H]
	\centering
	\includegraphics[width=0.7\linewidth]{image/lab4/04_4vic_bdf}
	\caption{Схема четырехразрядного вычитателя}
	\label{fig:044vicbdf}
\end{figure}


По описанному ранее порядку найдем временные задержки (рис. \ref{fig:044victime}).

\begin{figure}[H]
	\centering
	\includegraphics[width=0.7\linewidth]{image/lab4/04_4vic_time}
	\caption{Временные задержки четырехразрядного вычитателя}
	\label{fig:044victime}
\end{figure}

В результате моделирования временная диаграмма имеет следующий вид (рис. \ref{fig:044vicwvf}).

\begin{figure}[H]
	\centering
	\includegraphics[width=0.9\linewidth]{image/lab4/04_4vic_wvf}
	\caption{Временная диаграмма для четырехразрядного вычитателя}
	\label{fig:044vicwvf}
\end{figure}


Моделирование в режимах Functional и Time не отличается, т.к. на приведенной частоте задержек не видно.

Загрузим на плату и протестируем работу (рис. \ref{fig:044vicfoto})

\begin{figure}[H]
	\centering
	\includegraphics[width=0.7\linewidth]{image/lab4/2020-03-1413-07-27.JPG}
	\caption{Фото рабочей платы}
	\label{fig:044vicfoto}
\end{figure}

\section{Вывод}
В ходе проделанной работы был создан одноразрядный сумматор.
На его основе были спроектированы двухразрядный и четырехразрядный сумматоры, а также четырехразрядный вычитатель.
При помощи TimeQuest Timing Analysis были получены задержки для кажого входного параметра.
Схема была протестированы при помощи WaveForm, а также удалось загрузить схему на плату и протестировать работоспособность программы на плате.

\newpage 
\renewcommand{\refname}{{\normalsize СПИСОК ИСПОЛЬЗОВАННЫХ ИСТОЧНИКОВ}} 
\centering 
\begin{thebibliography}{9} 
	\addcontentsline{toc}{section}{\refname} 
	\bibitem{sql} Vijayakumar P., Vijayalakshmi V., Zayaraz G. Comparative study of hyperelliptic curve cryptosystem over prime field and its survey //International Journal of Hybrid Information Technology. – 2014. – Т. 7. – №. 1. – С. 137-146.
	\bibitem{sql} Антонов А., Филиппов А., Золотухо Р. Средства системной отладки САПР Quartus II //Компоненты и технологии. – 2008. – №. 89.
\end{thebibliography}

\end{document} % конец документа