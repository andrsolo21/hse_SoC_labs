%!TEX TS-program = xelatex

% Шаблон документа LaTeX создан в 2018 году
% Алексеем Подчезерцевым
% В качестве исходных использованы шаблоны
% 	Данилом Фёдоровых (danil@fedorovykh.ru) 
%		https://www.writelatex.com/coursera/latex/5.2.2
%	LaTeX-шаблон для русской кандидатской диссертации и её автореферата.
%		https://github.com/AndreyAkinshin/Russian-Phd-LaTeX-Dissertation-Template

\documentclass[a4paper,14pt]{article}

\input{data/preambular.tex}
\begin{document} % конец преамбулы, начало документа
\begin{titlepage}
	\begin{center}
		ПРАВИТЕЛЬСТВО РОССИЙСКОЙ ФЕДЕРАЦИИ \\
 		ФЕДЕРАЛЬНОЕ  ГОСУДАРСТВЕННОЕ АВТОНОМНОЕ \\
		ОБРАЗОВАТЕЛЬНОЕ УЧРЕЖДЕНИЕ ВЫСШЕГО ОБРАЗОВАНИЯ\\
		«НАЦИОНАЛЬНЫЙ ИССЛЕДОВАТЕЛЬСКИЙ УНИВЕРСИТЕТ\\
		«ВЫСШАЯ ШКОЛА ЭКОНОМИКИ»
	\end{center}
	
	\begin{center}
		\textbf{Московский институт электроники и математики}
		
		\textbf{Им. А.Н.Тихонова НИУ ВШЭ}
		
		\vspace{2ex}
		
		\textbf{Департамент компьютерной инженерии}
	\end{center}
	\vspace{1ex}	
	
	\vspace{1ex}
	\begin{center}
		\textbf{Практическая работа №1 \\
			«Знакомство с САПР Altera Quartus II» \\
			Вариант №13
	}
	\end{center}	

	\vspace{2ex}
	\vfill
	
	\vspace{2ex}
	
	\begin{flushright}
		\textbf{Выполнил:}
		
		\vspace{2ex}
		
		Студент группы БИВ174
		
		\vspace{2ex}
		
		Солодянкин Андрей Александрович
		
		\vspace{2ex}
		
		\textbf{Проверил:}
		
		\vspace{2ex}
		
		Романова Ирина Ивановна
	\end{flushright}

	\vspace{5ex}
	\begin{center}
		Москва \the\year \, г.
	\end{center}
	
\end{titlepage}
\addtocounter{page}{1}
\tableofcontents
\pagebreak

Известно длительность импульса ($\tau_u$ = 50 мск), длина волны МРЛС (λ = 0,1 м), мощность излучения (Р = 200 МВт), коэффициент направленного действия передающей антенны (G = 45000), минимальная энергия принимаемого сигнала (Э$_{пр.min} = 10^{-18}$),  ЭПР цели ( σ$_{ц}$= 0,01 м2).
\begin{figure}[H]
	\centering
	\includegraphics[width=1\linewidth]{screenshot001}
	\caption{}
	\label{fig:screenshot001}
\end{figure}

$$\text{Э}_c = P \cdot \tau_u$$

$$S_A = \dfrac{\lambda^2}{4\pi}\cdot G$$

$$D_{max} = \sqrt[4]{\dfrac{\text{Э}_c \cdot G \cdot S_A \cdot \sigma_{\text{ц}}}{(4\pi)^2 \cdot  \text{Э}_{np.min}}} $$

\pagebreak


$$\text{Э}_c = P \cdot \tau_u = 200 \cdot 10^{6} \text{ Вт} \cdot 50 \cdot 10^{-6} \text{ с} = 10^4 \text{ Дж}$$

$$S_A = \dfrac{\lambda^2}{4\pi}\cdot G = \dfrac{(0,1 \text{ м} )^2}{4\cdot 3,14}\cdot 45000 = 35,83\text{ м}^2$$

$$D_{max} = \sqrt[4]{\dfrac{\text{Э}_c \cdot G \cdot S_A \cdot \sigma_{\text{ц}}}{(4\pi)^2 \cdot  \text{Э}_{np.min}}} = 
\sqrt[4]{\dfrac{10^4 \text{ Дж} \cdot 45000 \cdot 35,83 \text{ м}^2 \cdot 0,01 \text{ м}^2}{(4 \cdot 3,14)^2 \cdot  10^{-18} \text{ Дж}}} = 
1,005 \cdot 10^6 \text{ м}
$$


\end{document} % конец документа