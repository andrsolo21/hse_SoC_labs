%!TEX TS-program = xelatex

% Шаблон документа LaTeX создан в 2018 году
% Алексеем Подчезерцевым
% В качестве исходных использованы шаблоны
% 	Данилом Фёдоровых (danil@fedorovykh.ru) 
%		https://www.writelatex.com/coursera/latex/5.2.2
%	LaTeX-шаблон для русской кандидатской диссертации и её автореферата.
%		https://github.com/AndreyAkinshin/Russian-Phd-LaTeX-Dissertation-Template

\documentclass[a4paper,14pt]{article}

\input{data/preambular.tex}
\begin{document} % конец преамбулы, начало документа
\begin{titlepage}
	\begin{center}
		ПРАВИТЕЛЬСТВО РОССИЙСКОЙ ФЕДЕРАЦИИ \\
 		ФЕДЕРАЛЬНОЕ  ГОСУДАРСТВЕННОЕ АВТОНОМНОЕ \\
		ОБРАЗОВАТЕЛЬНОЕ УЧРЕЖДЕНИЕ ВЫСШЕГО ОБРАЗОВАНИЯ\\
		«НАЦИОНАЛЬНЫЙ ИССЛЕДОВАТЕЛЬСКИЙ УНИВЕРСИТЕТ\\
		«ВЫСШАЯ ШКОЛА ЭКОНОМИКИ»
	\end{center}
	
	\begin{center}
		\textbf{Московский институт электроники и математики}
		
		\textbf{Им. А.Н.Тихонова НИУ ВШЭ}
		
		\vspace{2ex}
		
		\textbf{Департамент компьютерной инженерии}
	\end{center}
	\vspace{1ex}	
	
	\vspace{1ex}
	\begin{center}
		\textbf{Практическая работа №1 \\
			«Знакомство с САПР Altera Quartus II» \\
			Вариант №13
	}
	\end{center}	

	\vspace{2ex}
	\vfill
	
	\vspace{2ex}
	
	\begin{flushright}
		\textbf{Выполнил:}
		
		\vspace{2ex}
		
		Студент группы БИВ174
		
		\vspace{2ex}
		
		Солодянкин Андрей Александрович
		
		\vspace{2ex}
		
		\textbf{Проверил:}
		
		\vspace{2ex}
		
		Романова Ирина Ивановна
	\end{flushright}

	\vspace{5ex}
	\begin{center}
		Москва \the\year \, г.
	\end{center}
	
\end{titlepage}
\addtocounter{page}{1}
\tableofcontents
\pagebreak

\section{Задание}

Составить схему указанного выражения в базисе И, ИЛИ, НЕ.
Построить временную диаграмму и выполнить моделирование в режимах
Functional и Time. Оценить аппаратные ресурсы на реализацию схемы и
обосновать полученный результат. Упростить заданное логическое
выражение с помощью алгебры логики. Сравнить работу двух схем.
Запрограммировать учебную плату и продемонстрировать результаты
работы на макете (Часть 2).

13 Вариант
$$y =  !(x_1 U !(!x_1 \& !x_2))$$

\section{Выполнение работы}

Создаем проект.

Далее создадим блок схему:

\begin{figure}[H]
	\centering
	\includegraphics[width=0.7\linewidth]{image/01_01}
	\caption{Блок схема}
	\label{fig:0101}
\end{figure}

Таблица истинности:

\begin{table}[H]
	\caption{Таблица истинности}
	\centering
	\begin{tabular}{|c|c|c|}
		\hline
		$x_1$ & $x_2$ & $y$ \\ \hline
		0    & 0    & 1 \\ \hline
		0    & 1    & 0 \\ \hline
		1    & 0    & 0 \\ \hline
		1    & 1    & 0 \\ \hline
	\end{tabular}
\end{table}

Компилируем проект:

\begin{figure}[H]
	\centering
	\includegraphics[width=0.7\linewidth]{image/01_02}
	\caption{Результат компиляции}
	\label{fig:0102}
\end{figure}

Назначим пины:

\begin{figure}[H]
	\centering
	\includegraphics[width=0.7\linewidth]{image/01_03}
	\caption{Назначенные пины}
	\label{fig:0103}
\end{figure}

И проведем симуляцию:

\begin{figure}[H]
	\centering
	\includegraphics[width=0.7\linewidth]{image/01_04}
	\caption{Временные диаграммы}
	\label{fig:0104}
\end{figure}

Упростим логическое выражение:

$ !(x_1 U !(!x_1 \& !x_2)) = !(x_1 U x_1 U x_2) =  !(x_1 U x_2)$

$$ y =  !(x_1 U x_2)$$
 
Добавим к существующей схеме выход, соответствующий упрощенной функции(LEDR1).
 
\begin{figure}[H]
	\centering
	\includegraphics[width=0.7\linewidth]{image/01_07}
	\caption{Блок схема выражения}
	\label{fig:0107}
\end{figure}

Новая временная диаграмма выглядит следующим образом:

\begin{figure}[H]
	\centering
	\includegraphics[width=0.7\linewidth]{image/01_06}
	\caption{Временная диаграмма с упрощенной функцией}
	\label{fig:0106}
\end{figure}

 
\section{Вывод}
В ходе проделанной работы были построены логические блок схемы в базисе И, ИЛИ, НЕ, а также временные диаграммы полученных функций. Оценены аппаратные ресурсы схемы.


\newpage 
\renewcommand{\refname}{{\normalsize СПИСОК ИСПОЛЬЗОВАННЫХ ИСТОЧНИКОВ}} 
\centering 
\begin{thebibliography}{9} 
	\addcontentsline{toc}{section}{\refname} 
	\bibitem{sql} Vijayakumar P., Vijayalakshmi V., Zayaraz G. Comparative study of hyperelliptic curve cryptosystem over prime field and its survey //International Journal of Hybrid Information Technology. – 2014. – Т. 7. – №. 1. – С. 137-146.
	\bibitem{sql} Антонов А., Филиппов А., Золотухо Р. Средства системной отладки САПР Quartus II //Компоненты и технологии. – 2008. – №. 89.
\end{thebibliography}

\end{document} % конец документа